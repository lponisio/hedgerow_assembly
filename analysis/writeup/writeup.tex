\documentclass{article}\usepackage[]{graphicx}\usepackage[]{color}
%% maxwidth is the original width if it is less than linewidth
%% otherwise use linewidth (to make sure the graphics do not exceed the margin)
\makeatletter
\def\maxwidth{ %
  \ifdim\Gin@nat@width>\linewidth
    \linewidth
  \else
    \Gin@nat@width
  \fi
}
\makeatother

\definecolor{fgcolor}{rgb}{0.345, 0.345, 0.345}
\newcommand{\hlnum}[1]{\textcolor[rgb]{0.686,0.059,0.569}{#1}}%
\newcommand{\hlstr}[1]{\textcolor[rgb]{0.192,0.494,0.8}{#1}}%
\newcommand{\hlcom}[1]{\textcolor[rgb]{0.678,0.584,0.686}{\textit{#1}}}%
\newcommand{\hlopt}[1]{\textcolor[rgb]{0,0,0}{#1}}%
\newcommand{\hlstd}[1]{\textcolor[rgb]{0.345,0.345,0.345}{#1}}%
\newcommand{\hlkwa}[1]{\textcolor[rgb]{0.161,0.373,0.58}{\textbf{#1}}}%
\newcommand{\hlkwb}[1]{\textcolor[rgb]{0.69,0.353,0.396}{#1}}%
\newcommand{\hlkwc}[1]{\textcolor[rgb]{0.333,0.667,0.333}{#1}}%
\newcommand{\hlkwd}[1]{\textcolor[rgb]{0.737,0.353,0.396}{\textbf{#1}}}%

\usepackage{framed}
\makeatletter
\newenvironment{kframe}{%
 \def\at@end@of@kframe{}%
 \ifinner\ifhmode%
  \def\at@end@of@kframe{\end{minipage}}%
  \begin{minipage}{\columnwidth}%
 \fi\fi%
 \def\FrameCommand##1{\hskip\@totalleftmargin \hskip-\fboxsep
 \colorbox{shadecolor}{##1}\hskip-\fboxsep
     % There is no \\@totalrightmargin, so:
     \hskip-\linewidth \hskip-\@totalleftmargin \hskip\columnwidth}%
 \MakeFramed {\advance\hsize-\width
   \@totalleftmargin\z@ \linewidth\hsize
   \@setminipage}}%
 {\par\unskip\endMakeFramed%
 \at@end@of@kframe}
\makeatother

\definecolor{shadecolor}{rgb}{.97, .97, .97}
\definecolor{messagecolor}{rgb}{0, 0, 0}
\definecolor{warningcolor}{rgb}{1, 0, 1}
\definecolor{errorcolor}{rgb}{1, 0, 0}
\newenvironment{knitrout}{}{} % an empty environment to be redefined in TeX

\usepackage{alltt}
\usepackage{natbib}
\usepackage[unicode=true]{hyperref}
\usepackage{geometry}
\usepackage{hyperref}
\usepackage{color}
\usepackage{amsmath}
\usepackage{amssymb}
\usepackage{verbatim}
\usepackage{mathpazo}
\usepackage{setspace}
\usepackage{multirow}
\usepackage{fullpage}
\usepackage{lscape}
\usepackage{fancyhdr}
\usepackage{wrapfig,lipsum,booktabs}
\usepackage[normalem]{ulem}
\usepackage[parfill]{parskip}
\usepackage{multirow}
\geometry{tmargin=1in,bmargin=1in,lmargin=1in,rmargin=1in}

\bibliographystyle{rspublicnatsort}

%% for inline R code: if the inline code is not correctly parsed, you will see a message
\newcommand{\rinline}[1]{SOMETHING WRONG WITH knitr}


\IfFileExists{upquote.sty}{\usepackage{upquote}}{}
\begin{document}
\title{Diversification practices reduce organic to conventional yield
gap: A walkthrough of the anlaysis}
\author{Lauren Ponisio}

\maketitle
\section{Overview}
\label{sec:overview}


In our study
\href{http://nature.berkeley.edu/~lponisio/wp-content/uploads/2014/08/Ponisio2014.pdf}{\textcolor{blue}
{Ponisio
et al.~2015}} we compared the yields or organic and conventional
agriculture using a custom-build meta-analytic model. Here I explain
the analytic differences between our study and others, example our
modeling decisions, and walk through our code. Below is the output of
the final meta-analytic model, which will be described in detail in
this document.

\begin{knitrout}
\definecolor{shadecolor}{rgb}{0.969, 0.969, 0.969}\color{fgcolor}\begin{kframe}
\begin{alltt}
\hlkwd{print}\hlstd{(out}\hlopt{$}\hlstd{bugs,} \hlkwc{dig}\hlstd{=}\hlnum{3}\hlstd{)}
\end{alltt}
\begin{verbatim}
## Inference for Bugs model at "model.jags", fit using jags,
##  3 chains, each with 1e+05 iterations (first 100 discarded), n.thin = 10
##  n.sims = 29970 iterations saved
##            mu.vect sd.vect      2.5%       25%       50%       75%
## cv.obs       1.154   0.134     0.909     1.061     1.149     1.241
## exp.mu       0.808   0.019     0.771     0.795     0.808     0.821
## sigma        0.189   0.023     0.145     0.172     0.188     0.204
## deviance -1293.537  36.474 -1362.740 -1318.528 -1294.309 -1269.421
##              97.5%  Rhat n.eff
## cv.obs       1.434 1.001  5100
## exp.mu       0.846 1.001 27000
## sigma        0.236 1.001  9400
## deviance -1219.532 1.001 20000
## 
## For each parameter, n.eff is a crude measure of effective sample size,
## and Rhat is the potential scale reduction factor (at convergence, Rhat=1).
## 
## DIC info (using the rule, pD = var(deviance)/2)
## pD = 665.1 and DIC = -628.4
## DIC is an estimate of expected predictive error (lower deviance is better).
\end{verbatim}
\end{kframe}
\end{knitrout}

\section{Where it all began, Seufert et al.~2011}
\label{sec:beginning}

How organic agriculture may contribute to world food production has
been subject to vigorous debate over the past decade. Early reviews
comparing organic to conventional agriculture found yield gaps of
$8-9$\% in developed countries \citep{Stanhill1990, Badgley2007} but
yield gains of as much as $180$\% in developing countries. Two recent
meta-analyses, however, found organic yields to be $20-25$\% lower
than conventional yields \citep{dePonti2012, Seufert2012}. These
studies used different criteria for selecting the data to be included,
but importantly each of the above studies used different analytical
methods to combine the data across the different sub-studies. The
studies comparing organic and conventional yields systems often
reported yields from multiple crops across several years. In addition,
they tended to compare multiple treatments (usually organic) to one
control treatment (usually conventional). For example from
\cite{denison2004crop}:

\begin{knitrout}
\definecolor{shadecolor}{rgb}{0.969, 0.969, 0.969}\color{fgcolor}\begin{kframe}
\begin{alltt}
\hlstd{samp}
\end{alltt}
\begin{verbatim}
##       study   crop conv org year
## 100 study13  maize  CMT OMT 2001
## 101 study13  maize  CMT OMT 1999
## 102 study13 tomato  CWT OMT 2002
## 103 study13  maize  LMT OMT 2000
## 104 study13  maize  CMT OMT 1996
## 105 study13  maize  LMT OMT 2001
## 106 study13 tomato  LMT OMT 1999
## 107 study13  maize  CMT OMT 1997
## 108 study13 tomato  CMT OMT 2001
## 109 study13 tomato  CMT OMT 1995
## 110 study13 tomato  LMT OMT 1995
## 111 study13  maize  LMT OMT 1999
## 112 study13  maize  LMT OMT 2002
## 113 study13  maize  LMT OMT 1994
## 114 study13  maize  CMT OMT 1994
## 115 study13  maize  CMT OMT 2002
## 116 study13 tomato  LMT OMT 1998
## 117 study13 tomato  CMT OMT 1994
## 118 study13 tomato  CMT OMT 1999
## 119 study13 tomato  CWT OMT 1994
## 120 study13 tomato  CWT OMT 1995
## 121 study13 tomato  CMT OMT 1996
## 122 study13 tomato  LMT OMT 1996
## 123 study13 tomato  CWT OMT 1996
## 124 study13  maize  LMT OMT 1995
## 125 study13  maize  CMT OMT 1995
## 126 study13  maize  LMT OMT 1996
## 127 study13 tomato  LMT OMT 1994
## 128 study13 tomato  CWT OMT 1998
## 129 study13 tomato  CMT OMT 1998
## 130 study13 tomato  CMT OMT 1997
## 131 study13 tomato  LMT OMT 1997
## 132 study13 tomato  LMT OMT 2001
## 133 study13 tomato  CWT OMT 2001
## 134 study13 tomato  CWT OMT 1997
## 135 study13  maize  LMT OMT 1997
\end{verbatim}
\end{kframe}
\end{knitrout}

Using the organic to conventional yield comparisons without taking
into account the underlying data structure can lead to potential
pseudo-replication and an understated Type 1 error rate. We used a
randomization test to estimate the Type I error rate of the Seufert et
al.~analysis.  We forced the null hypothesis to be true by randomly
re-assigning the 'organic' and 'conventional' labels for each study
and then using the R package {\tt Metafor} \citep{metafor} to
implement a random effects meta-analysis on each randomized dataset.
Repeating this procedure $10^5$ times enabled us to determine the Type
I error rate (false rejection) resulting from not accounting for the
hierarchical structure of the data.  In over $50\%$ of simulations,
the null hypothesis was rejected using a nominal Type I error rate of
0.05 (Fig.~\ref{fig:permutation}).  In other words, even if organic
and conventional yields are known not to be different, applying the
model used by \cite{Seufert2012} for these data would lead to the
conclusion that they are significantly different in over $50\%$ of
cases. This means that the actual Type I error rate is inflated
relative to what was reported, leading to the following related
statements: the significance levels were overstated; the confidence
intervals were underestimated; the uncertainty was not fully accounted
for.

\begin{figure} \centering
\includegraphics[width=0.5\textwidth]{figure/permutation.pdf}
\caption{The distribution of $p$-values when the null hypothesis was
forced to be true using the data and analysis type present in
Seufert et al.\  \cite{Seufert2012}.  If the analysis procedure was
valid for these data, the distribution of P-values should be
uniform between 0 and 1.  Instead it is sharply shifted toward low
P-values.  In over $50\%$ of simulations, the null hypothesis was
rejected using a nominal Type I error rate of 0.05 (red region
above).}
\label{fig:permutation}
\end{figure}

The de Ponti et al.\ \citep{dePonti2012} study had similar issues with
pseudo-replication. Additionally they did not account for the sampling
variance within studies, which is the recommended practice to deal
with unequal variances in the sample of studies \citep{Gurevitch1999}.

Given these methodological and data-related critiques, a new study was
needed to produce a more robust estimate of the gap between organic
and conventional yields.  We developed a hierarchical meta-analytic
framework that overcomes the methodological pitfalls of previous
studies by accounting for both the multi-level nature of the data and
the yield variation within studies.  Furthermore, via a literature
search we compiled a more extensive and up-to-date meta-dataset,
comprising 1071 organic to conventional yield comparisons from 115
studies --- over three times the number of observations of any of the
previous analyses. Our meta-dataset includes studies from 38 countries
and 52 crop species over a span of 35 years.

\section{Building our meta-analytic model}
\label{sec:model-building}

We built a hierarchical meta-analytic model to generate an estimate of
the yield gap. Following standard practice, we compared the natural
log of the ratios between organic and conventional yields (the
``response ratio'') across studies \citep{Hedges1999,
Seufert2012}. The response ratio is more normally distributed than the
raw ratio and independent of the units of measurement used within a
study and, thus, comparable across studies \citep{Hedges1999}.

\begin{knitrout}
\definecolor{shadecolor}{rgb}{0.969, 0.969, 0.969}\color{fgcolor}\begin{kframe}
\begin{alltt}
\hlkwd{plot}\hlstd{(}\hlkwd{density}\hlstd{(meta.dat}\hlopt{$}\hlstd{lnR),} \hlkwc{main}\hlstd{=}\hlstr{"Distribution of log response ratios"}\hlstd{)}
\end{alltt}
\end{kframe}
\includegraphics[width=\maxwidth]{figure/RR_density-1} 

\end{knitrout}

We constructed a hierarchical regression model to account for
the dependencies in the yield data. We expanded on the traditional
random effects model \citep{Hedges1985} by considering three
additional sources of random variation (i.e., random effects): 1)
between studies, 2) within a study between years, and 3) within a year
between response ratios (e.g., across replicated trials of a crop
planted at different times in the season). We also considered whether
the variances of the random effect distributions for 2) and 3) were
shared across studies, or study-specific.

The full possible model, prior to model selection, with all sources of
random variation is

\begin{equation}
  \begin{split}
    \label{eq:fullModel}
    y_{ijk} &= \mu + \alpha_i + \beta_{ij} +  \eta_{ijk} +
    \epsilon_{ijk}  \\
    \alpha_{i} &\sim N(0, \sigma^2_{\alpha})\\
    \beta_{ij} &\sim N(0, \sigma^2_{\beta}[i])\\
    \eta_{ijk} &\sim N(0, \sigma^2_{\eta}[i])\\
    \epsilon_{ijk} &\sim N(0, S_{ijk}])\\
    \sigma^2_{\beta}[i] &\sim \Gamma(CV_\beta, scale_\beta)\\
    \sigma^2_{\eta}[i] &\sim \Gamma(CV_\eta, scale_\eta)
  \end{split}
\end{equation}
% where $y_{ijk}$ is the observed magnitude of the $k^{\mathrm{th}}$
response ratio from the $j^{\mathrm{th}}$ year of the
$i^{\mathrm{th}}$ study, $\mu$ is the mean response ratio across
studies, $\alpha_{i}$ is the effect of $i^{\mathrm{th}}$ study,
$\beta_{ij}$ is the effect of $j^{\mathrm{th}}$ year of the
$i^{\mathrm{th}}$ study, $\eta_{ijk}$ is the effect of the
$k^{\mathrm{th}}$ response ratio of the $j^{\mathrm{th}}$ year of
$i^{\mathrm{th}}$ study, and $\epsilon_{ijk}$ is the
residual. $\sigma^2_{\alpha}$ is the between study variance,
$\sigma^2_{\beta}[i]$ is the between year variance of study $i$,
$\sigma^2_{\eta}[i]$ is the within year, between response ratio
variance of study $i$, and $S_{ijk}$ is the variance of response ratio
$ijk$ as reported by its study. $CV_\beta$ and $CV_\eta$ and
$scale_\beta$ and $scale_\eta$ are the coefficient of variation and
scale parameters of the gamma distributions of the study-specific
between- and within-year variances.  When response ratios that shared
a common control were combined, $y_{ijk}$ corresponds to the aggregate
within-study response ratio \citep[Eq.~3,][]{Lajeunesse2011} and
$S_{ijk}$ is its pooled variance \citep[Eq.~8,][]{Lajeunesse2011}.

\subsection{Parameter inclusion}
To determine the levels of hierarchy supported by the data, we
sequentially added random effects and examined the posteriors of the
parameters to determine the support for their inclusion. We also
confirmed our selection with Deviance Information Criterion (DIC). The
DIC can be problematic for hierarchical models because the effective
number of parameters is not clearly defined \citep{gelman2006data,
kery2012bayesian}. The DIC was therefore used in combination with a
visual examination of the posterior distributions of the parameters to
select the best supported model.

For the variance within and between year random effect distributions,
we considered two parameterizations: 1) the variance terms, denoted
$\sigma^2_{\eta}$ and $\sigma^2_{\beta}$, respectively, were shared
across all studies each with a Uniform(0,100) prior, and 2) the
variance terms were study-specific (i.e., $\sigma^2_{\eta}[i]$ and
$\sigma^2_{\beta}[i]$). In the latter case, the study-specific
precision terms (1/variance) were assumed to be distributed according
to a gamma distribution whose parameters were
estimated. Uniform(0,100) priors were used for the coefficient of
variation ($1/\sqrt{shape}$) and the square root of the scale.

We first added a random effect of study and examined the posterior for
$\sigma_{\alpha}$ (the standard deviation of the common distribution
from which the study effects are drawn). The posterior was clearly
differentiated from zero (Fig.~\ref{fig:posteriors}a). We next added
random variation within a year and examined $\sigma_{\eta}$. We found
it was also clearly different from zero
(Fig.~\ref{fig:posteriors}b). The DIC was also smaller than when only
a random effect of study was included (Tab.~\ref{table:params}). Next
we allowed the within-year precisions to be study-specific and follow
a gamma distribution. We examined the coefficient of variation of the
gamma distribution and found it was clearly differentiated from zero
(Fig.~\ref{fig:posteriors}c). The DIC was also smaller than when a
single within year effect was shared across studies
(Tab.~\ref{table:params}). Lastly, we added a between year random
effect and examined $\sigma_{\beta}$. The posterior was concentrated
at zero (Fig.~\ref{fig:posteriors}d) so we concluded there was
insufficient support for including it in the model. The estimate of
the yield gap and its uncertainty did not differ substantially from
when no between year effect was included
(Fig.~\ref{fig:model_selection}), the DIC, however, was marginally
smaller then when no between year random effect was included
(Tab.~\ref{table:params}).

\clearpage

\begin{figure}
\centering
\includegraphics[width=0.8\textwidth]{figure/posteriors.pdf}
\caption{The posterior distributions for the random effect of a)
study ($\sigma_{\alpha}$); b) response ratios within a year
($\sigma_{\eta}$); c) response ratios within a year where the
within year variance is study-specific, $CV_{\sigma_{\eta}}$ is
the coefficient of variation ($1/\sqrt{shape}$) of the gamma
distribution (this model is most supported by the data); and d)
between year ($\sigma_{\beta}$). Including a between-year variance
term was not supported by the data (the posterior for
($\sigma_{\beta}$ is not differentiated from zero ).}
\label{fig:posteriors}
\end{figure}
\clearpage

\begin{figure}
\centering
\includegraphics[width=0.8\textwidth]{figure/model_selection.pdf}
\caption{The sensitivity of the yield gap to including different
levels of hierarchy in the model. The random effects included in
the model are: a) study ($\sigma_{\alpha}$); b) study and response
ratios within a year ($\sigma_{\eta}$); c) study and response
ratios within a year where the within year variance is
study-specific ($\sigma_{\eta}[i]$) (this model is most supported
by the data); and d) study, study-specific within-year variance,
and between year ($\sigma_{\beta}$). Including a between-year
variance term was not supported by the data (the posterior for
($\sigma_{\beta}$ is not differentiated from zero ). Values are
the posterior mean with 95\% credible intervals.}
\label{fig:model_selection}
\end{figure}
\clearpage

\begin{table}
  \renewcommand*\arraystretch{1.25}
  \centering
  \caption{Parameter posteriors for models without explanatory
    variables. $\mu$ is the true mean response ratio across years and
    studies, $\sigma_{\alpha}$ is the standard deviation of the
    distribution from which the study random effects are drawn;
    $\sigma_{\eta}$ is the standard deviation of the distribution from
    which the within year random effects are drawn; $CV_{\sigma_\eta}$ is the
    coefficient of variation of the gamma from which the
    study-specific within-year variance are drawn; and $\sigma_\beta$
    is the standard deviation of the distribution of random between year
    effects. Values of Rhat $< 1.1$ indicate convergence. Lower
    Deviance Information Criterion (DIC) indicates better model fit to
    the data.}

  \vspace{20pt}
  \label{table:params}
  \begin{tabular}{|c|c|c|c|c|} \hline
    Parameter & Posterior mean & Posterior standard deviation
    & 95\% CI & Rhat \\ \hline
    \multicolumn{5}{|l|}{Study random effect, DIC=1684.8} \\ \hline
    \multirow{1}{*}{$\mu$} &
    $0.795$ & $0.027$ & $0.742 - 0.848$ & $1.001$ \\ \hline
    \multirow{1}{*}{$\sigma_{\alpha}$} &
    $0.341$ & $0.026$ & $0.294 - 0.396$ & $1.001$ \\ \hline

    \multicolumn{5}{|l|}{Study and within year random effects,
      DIC= -565.9} \\ \hline
    \multirow{1}{*}{$\mu$} &
    $0.788$ & $0.021$ & $0.749 - 0.829$ & $1.001$ \\ \hline
    \multirow{1}{*}{$\sigma_{\alpha}$} &
    $0.188$ & $0.024$ & $0.144 - 0.239$ & $1.001$ \\ \hline
    \multirow{1}{*}{$\sigma_{\eta}$} &
    $0.312$ & $0.011$ & $0.291 - 0.333$ & $1.001$ \\ \hline

    \multicolumn{5}{|l|}{Study and study-specific within year random
      effects, DIC= -618.0} \\ \hline
    \multirow{1}{*}{$\mu$} &
    $0.808$ & $0.019$ & $0.771 - 0.845$ & $1.001$ \\ \hline
    \multirow{1}{*}{$\sigma_{\alpha}$} &
    $0.189$ & $0.023$ & $0.145 - 0.237$ & $1.001$ \\ \hline
   \multirow{1}{*}{$CV_{\sigma_\eta}$} &
    $1.155$ & $0.135$ & $0.907 - 1.436$ & $1.001$ \\ \hline

    \multicolumn{5}{|l|}{Study, study-specific within year,
      and between year random effects, DIC= -621.2} \\ \hline
     \multirow{1}{*}{$\mu$} &
    $0.808$ & $0.019$ & $0.770 - 0.846$ & $1.001$ \\ \hline
    \multirow{1}{*}{$\sigma_{\alpha}$} &
    $0.186$ & $0.024$ & $0.142 - 0.234$ & $1.001$ \\ \hline
   \multirow{1}{*}{$CV_{\sigma_\eta}$} &
    $1.157$ & $0.136$ & $0.907 - 1.440$ & $1.001$ \\ \hline
   \multirow{1}{*}{$\sigma_{\beta}$} &
    $0.041$ & $0.027$ & $0.002 - 0.098$ & $1.001$ \\ \hline

  \end{tabular}
\end{table}
\clearpage

The best supported model, coded in JAGS, is

\begin{knitrout}
\definecolor{shadecolor}{rgb}{0.969, 0.969, 0.969}\color{fgcolor}\begin{kframe}
\begin{verbatim}
## model {
##      for(study in 1:Nstudy) {
##        for(year in 1:Nyear[study]) {
##          for(obs in 1:Nobs[study,year]) {
##            P[study,year,obs] <- 1/V[study,year,obs]
##            RR[study,year,obs] ~ dnorm(mu.RR[study,year,obs],
##                                       P[study,year,obs])
##            mu.RR[study,year,obs] ~ dnorm(mu.study[study],
##                                      tau.yr.obs[study])
##          }
##        }
##        mu.study[study] ~ dnorm(mu, tau)
##        tau.yr.obs[study] ~ dgamma(shape.obs, scale.obs)
##      }
## 
##      mu ~ dnorm(0, 1e-4)
##      exp.mu <- exp(mu)
##      tau <- 1 / (sigma * sigma)
##      sigma ~ dunif(0, 100)
## 
##      shape.obs <- (1/cv.obs)^2
##      cv.obs ~ dunif(0, 100)
##      scale.obs <- (1/in.scale.obs)^2
##      in.scale.obs ~ dunif(0, 100)
##    }
\end{verbatim}
\end{kframe}
\end{knitrout}

Where the parameters names match the notation in
Equ. \ref{eq:fullModel}.

\section{Analysis}
\label{sec:analysis}

After deciding on the model parameterization, the most difficult part
of running a model in JAGS in getting tha data in the right format (in
my opinion). Our situation was particularly difficult because we
needed to build in flexibility to pool different combinations of
response ratios using the method presented in Lajeunesse
\cite{Lajeunesse2011} depending on the explanatory variable being
considered (when response ratios that share a control split between
different levels of an explanatory variable, they can be left
un-pooled). Below I focus on the best supported model without
explanatory variables.

First we have a function to calculate pooled RR and their variances
\begin{knitrout}
\definecolor{shadecolor}{rgb}{0.969, 0.969, 0.969}\color{fgcolor}\begin{kframe}
\begin{alltt}
\hlcom{## applies the Lajeunesse (2011) method for combining RR with shared}
\hlcom{## controls. Returns a pooled RR and variance}
\hlstd{poolCommonControl} \hlkwb{<-} \hlkwa{function}\hlstd{(}\hlkwc{dd}\hlstd{) \{}

  \hlcom{## function to make covariance matrix:}
  \hlstd{covMat} \hlkwb{<-} \hlkwa{function}\hlstd{(}\hlkwc{x}\hlstd{)\{}
    \hlcom{## case 1: organic is repeated}
    \hlkwa{if}\hlstd{(}\hlkwd{all}\hlstd{(}\hlkwd{diff}\hlstd{(}\hlkwd{as.numeric}\hlstd{(x[,}\hlstr{"mean.org"}\hlstd{]))}\hlopt{==}\hlnum{0}\hlstd{))}
      \hlstd{vals} \hlkwb{<-} \hlkwd{as.numeric}\hlstd{(x[}\hlnum{1}\hlstd{,} \hlkwd{c}\hlstd{(}\hlstr{"sd.org"}\hlstd{,} \hlstr{"n.organic"}\hlstd{,} \hlstr{"mean.org"}\hlstd{)])}
    \hlkwa{if}\hlstd{(}\hlkwd{all}\hlstd{(}\hlkwd{diff}\hlstd{(}\hlkwd{as.numeric}\hlstd{(x[,}\hlstr{"mean.conv"}\hlstd{]))}\hlopt{==}\hlnum{0}\hlstd{))}
      \hlstd{vals} \hlkwb{<-} \hlkwd{as.numeric}\hlstd{(x[}\hlnum{1}\hlstd{,} \hlkwd{c}\hlstd{(}\hlstr{"sd.conv"}\hlstd{,} \hlstr{"n.conv"}\hlstd{,} \hlstr{"mean.conv"}\hlstd{)])}

    \hlstd{m} \hlkwb{<-} \hlkwd{matrix}\hlstd{(vals[}\hlnum{1}\hlstd{]}\hlopt{^}\hlnum{2}\hlopt{/}\hlstd{(vals[}\hlnum{2}\hlstd{]}\hlopt{*}\hlstd{vals[}\hlnum{3}\hlstd{]}\hlopt{^}\hlnum{2}\hlstd{),}
                \hlkwc{ncol}\hlstd{=}\hlkwd{nrow}\hlstd{(dd),} \hlkwc{nrow}\hlstd{=}\hlkwd{nrow}\hlstd{(dd))}
    \hlkwd{diag}\hlstd{(m)} \hlkwb{<-} \hlkwd{as.numeric}\hlstd{(x[,}\hlstr{"varlnR"}\hlstd{])}
    \hlstd{m}
  \hlstd{\}}

  \hlcom{## covariance matrix and its inverse:}
  \hlstd{V} \hlkwb{<-} \hlkwd{covMat}\hlstd{(dd)}
  \hlstd{V.inv} \hlkwb{<-} \hlkwd{solve}\hlstd{(V)}

  \hlstd{X} \hlkwb{<-} \hlkwd{matrix}\hlstd{(}\hlnum{1}\hlstd{,} \hlkwd{nrow}\hlstd{(V))}
  \hlstd{E} \hlkwb{<-} \hlkwd{matrix}\hlstd{(dd[,}\hlstr{"lnR"}\hlstd{],} \hlkwd{nrow}\hlstd{(V))}
  \hlcom{## compute and return RR_bar and variance(RR_bar), as defined in}
  \hlcom{## Eq. 3 and subsequent text in Lajeunesse (2011):}
  \hlkwd{return}\hlstd{(}\hlkwd{c}\hlstd{(}\hlkwc{obs}\hlstd{=dd}\hlopt{$}\hlstd{obs[}\hlnum{1}\hlstd{],}
           \hlkwc{lnR}\hlstd{=}\hlkwd{as.numeric}\hlstd{(}\hlkwd{solve}\hlstd{(}\hlkwd{t}\hlstd{(X)} \hlopt \hlstd{V.inv} \hlopt \hlstd{X)} \hlopt
                          \hlstd{(}\hlkwd{t}\hlstd{(X)} \hlopt \hlstd{V.inv} \hlopt \hlstd{E)),}
           \hlkwc{varlnR}\hlstd{=}\hlkwd{as.numeric}\hlstd{(}\hlkwd{solve}\hlstd{(}\hlkwd{t}\hlstd{(X)} \hlopt \hlstd{V.inv} \hlopt \hlstd{X))))}
\hlstd{\}}
\end{alltt}
\end{kframe}
\end{knitrout}

Which is called by this function which combines the meta-data based on
keys

\begin{knitrout}
\definecolor{shadecolor}{rgb}{0.969, 0.969, 0.969}\color{fgcolor}\begin{kframe}
\begin{alltt}
\hlcom{## takes raw meta-data and a vector of covariates and combines data}
\hlcom{## using lajeunesse (2011). Returns a dataframe with keys, the}
\hlcom{## observation number, the pooled RR and its variance}

\hlstd{poolRR} \hlkwb{<-} \hlkwa{function}\hlstd{(}\hlkwc{D}\hlstd{,} \hlkwc{covariates}\hlstd{) \{}
  \hlcom{## first split the data up into relevant groups:}
  \hlkwa{if}\hlstd{(}\hlkwd{length}\hlstd{(covariates)} \hlopt{==} \hlnum{1}\hlstd{) \{}
    \hlstd{dd} \hlkwb{<-} \hlkwd{split}\hlstd{(D,} \hlkwd{paste}\hlstd{(D}\hlopt{$}\hlstd{study,}
                         \hlstd{D[,covariates],}
                         \hlstd{D}\hlopt{$}\hlstd{year_conv,}
                         \hlstd{D}\hlopt{$}\hlstd{mult_yrs,}
                         \hlstd{D}\hlopt{$}\hlstd{duplicate_trt_dat,}
                         \hlkwc{sep}\hlstd{=}\hlstr{';'}\hlstd{))}
  \hlstd{\}}
  \hlkwa{if}\hlstd{(}\hlkwd{length}\hlstd{(covariates)} \hlopt{>} \hlnum{1}\hlstd{) \{}
    \hlstd{dd} \hlkwb{<-} \hlkwd{split}\hlstd{(D,} \hlkwd{paste}\hlstd{(D}\hlopt{$}\hlstd{study,}
                         \hlkwd{apply}\hlstd{(D[,covariates],} \hlnum{1}\hlstd{, paste,} \hlkwc{collapse}\hlstd{=}\hlstr{':'}\hlstd{),}
                         \hlstd{D}\hlopt{$}\hlstd{year_conv,}
                         \hlstd{D}\hlopt{$}\hlstd{mult_yrs,}
                         \hlstd{D}\hlopt{$}\hlstd{duplicate_trt_dat,}
                         \hlkwc{sep}\hlstd{=}\hlstr{';'}\hlstd{))}
  \hlstd{\}}

  \hlcom{## combine response ratios for studies that share treatments using}
  \hlcom{## the method given in Lajeunesse (2011):}
  \hlstd{share.trts} \hlkwb{<-} \hlkwd{which}\hlstd{(}\hlkwd{sapply}\hlstd{(}\hlkwd{strsplit}\hlstd{(}\hlkwd{names}\hlstd{(dd),} \hlkwc{split}\hlstd{=}\hlstr{';'}\hlstd{),}
                             \hlkwa{function}\hlstd{(}\hlkwc{x}\hlstd{)} \hlkwd{rev}\hlstd{(x)[[}\hlnum{1}\hlstd{]])} \hlopt{==} \hlstr{'yes'}\hlstd{)}

  \hlstd{dd[share.trts]} \hlkwb{<-} \hlkwd{lapply}\hlstd{(dd[share.trts],}
                           \hlstd{poolCommonControl)}

  \hlstd{dd[}\hlopt{-}\hlstd{share.trts]} \hlkwb{<-}
    \hlkwd{lapply}\hlstd{(dd[}\hlopt{-}\hlstd{share.trts],} \hlkwa{function}\hlstd{(}\hlkwc{x}\hlstd{) x[,}\hlkwd{c}\hlstd{(}\hlstr{'obs'}\hlstd{,} \hlstr{'lnR'}\hlstd{,} \hlstr{'varlnR'}\hlstd{)])}
  \hlkwd{return}\hlstd{(}\hlkwd{do.call}\hlstd{(rbind, dd))}
\hlstd{\}}
\end{alltt}
\end{kframe}
\end{knitrout}

This is all brought together by this long, unwieldy function that
returns the data structure necessary for the JAGS model including the
indexes for looping and arrays of RR, variances and covariates is
applicable

\begin{knitrout}
\definecolor{shadecolor}{rgb}{0.969, 0.969, 0.969}\color{fgcolor}\begin{kframe}
\begin{alltt}
\hlcom{## makes data arrays for Bayesian analysis; takes the raw}
\hlcom{## meta-data set, whether to apply the lajeunesse method to combine RR}
\hlcom{## with shared controls, and a vector of explanatory variables. The}
\hlcom{## default for covariates is "Crop.species" so RR are never combined}
\hlcom{## between crop species. Returns a list of the number of studies}
\hlcom{## (Nstudy), the number of years for each study (Nyear), a matrix}
\hlcom{## (study, year) of the number of observations in each year (Nobs),}
\hlcom{## and an array (study, year, observation) of the RR and their}
\hlcom{## variances, and an array of covariates if applicable}

\hlstd{makeData} \hlkwb{<-} \hlkwa{function}\hlstd{(}\hlkwc{meta.dat}\hlstd{,}
                     \hlkwc{lajeunesse} \hlstd{=} \hlnum{TRUE}\hlstd{,}
                     \hlkwc{covariates}\hlstd{=}\hlkwd{c}\hlstd{(}\hlstr{'Crop.species'}\hlstd{)) \{}

  \hlstd{convert2Int} \hlkwb{<-} \hlkwa{function}\hlstd{(}\hlkwc{x}\hlstd{)} \hlkwd{seq_along}\hlstd{(x)[}\hlkwd{match}\hlstd{(x,} \hlkwd{unique}\hlstd{(x))]}

  \hlkwa{if}\hlstd{(lajeunesse) \{}
    \hlstd{dd} \hlkwb{<-} \hlkwd{poolRR}\hlstd{(meta.dat,}
                 \hlkwc{covariates}\hlstd{=covariates)}
    \hlstd{keys.split} \hlkwb{<-} \hlkwd{sapply}\hlstd{(}\hlkwd{rownames}\hlstd{(dd), strsplit,} \hlkwc{split}\hlstd{=}\hlstr{';'}\hlstd{)}
    \hlstd{get} \hlkwb{<-} \hlkwa{function}\hlstd{(}\hlkwc{i}\hlstd{)} \hlkwd{as.vector}\hlstd{(}\hlkwd{sapply}\hlstd{(keys.split,} \hlkwa{function}\hlstd{(}\hlkwc{x}\hlstd{) x[i]))}
    \hlstd{study} \hlkwb{<-} \hlkwd{get}\hlstd{(}\hlnum{1}\hlstd{)}
    \hlstd{year} \hlkwb{<-} \hlkwd{get}\hlstd{(}\hlnum{3}\hlstd{)}
  \hlstd{\}} \hlkwa{else} \hlstd{\{}
    \hlstd{dd} \hlkwb{<-} \hlstd{meta.dat}
    \hlstd{study} \hlkwb{<-} \hlkwd{as.character}\hlstd{(meta.dat}\hlopt{$}\hlstd{study)}
    \hlstd{year} \hlkwb{<-} \hlstd{meta.dat}\hlopt{$}\hlstd{year_conv}
  \hlstd{\}}

  \hlcom{## make study index}
  \hlstd{ind.study} \hlkwb{<-} \hlkwd{convert2Int}\hlstd{(study)}

  \hlcom{## make year index}
  \hlstd{ind.year} \hlkwb{<-} \hlkwd{rep}\hlstd{(}\hlnum{NA}\hlstd{,} \hlkwd{nrow}\hlstd{(dd))}
  \hlkwa{for}\hlstd{(i} \hlkwa{in} \hlkwd{unique}\hlstd{(ind.study))}
    \hlstd{ind.year[ind.study}\hlopt{==}\hlstd{i]} \hlkwb{<-} \hlkwd{convert2Int}\hlstd{(year[ind.study}\hlopt{==}\hlstd{i])}

  \hlcom{## make obs index}
  \hlstd{ind.obs} \hlkwb{<-} \hlkwd{rep}\hlstd{(}\hlnum{NA}\hlstd{,} \hlkwd{nrow}\hlstd{(dd))}
  \hlkwa{for}\hlstd{(i} \hlkwa{in} \hlkwd{unique}\hlstd{(ind.study))}
    \hlkwa{for}\hlstd{(j} \hlkwa{in} \hlkwd{unique}\hlstd{(ind.year[ind.study}\hlopt{==}\hlstd{i]))}
      \hlstd{ind.obs[ind.year}\hlopt{==}\hlstd{j} \hlopt{&} \hlstd{ind.study}\hlopt{==}\hlstd{i]} \hlkwb{<-}
        \hlkwd{seq_len}\hlstd{(}\hlkwd{sum}\hlstd{(ind.year}\hlopt{==}\hlstd{j} \hlopt{&} \hlstd{ind.study}\hlopt{==}\hlstd{i))}

  \hlcom{## make response ratio and vi matrices}
  \hlstd{makeMat} \hlkwb{<-} \hlkwa{function}\hlstd{(}\hlkwc{d.vec}\hlstd{) \{}
    \hlstd{mat} \hlkwb{<-} \hlkwd{array}\hlstd{(}\hlnum{NA}\hlstd{,} \hlkwc{dim}\hlstd{=}\hlkwd{c}\hlstd{(}\hlkwd{max}\hlstd{(ind.study),}
                           \hlkwd{max}\hlstd{(ind.year),}
                           \hlkwd{max}\hlstd{(ind.obs)))}
    \hlstd{mat[}\hlkwd{cbind}\hlstd{(ind.study, ind.year, ind.obs)]} \hlkwb{<-} \hlstd{d.vec}
    \hlkwd{return}\hlstd{(mat)}
  \hlstd{\}}
  \hlstd{RR.mat} \hlkwb{<-} \hlkwd{makeMat}\hlstd{(dd}\hlopt{$}\hlstd{lnR)}
  \hlstd{vi.mat} \hlkwb{<-} \hlkwd{makeMat}\hlstd{(dd}\hlopt{$}\hlstd{varlnR)}

  \hlcom{## make index bounds}
  \hlstd{nstudy} \hlkwb{<-} \hlkwd{max}\hlstd{(ind.study)}
  \hlstd{nyear}  \hlkwb{<-} \hlkwd{apply}\hlstd{(RR.mat[,,}\hlnum{1}\hlstd{],}  \hlnum{1}\hlstd{,} \hlkwa{function}\hlstd{(}\hlkwc{x}\hlstd{)} \hlkwd{sum}\hlstd{(}\hlopt{!}\hlkwd{is.na}\hlstd{(x)))}
  \hlstd{nobs}   \hlkwb{<-} \hlkwd{apply}\hlstd{(RR.mat[,,],} \hlnum{1}\hlopt{:}\hlnum{2}\hlstd{,} \hlkwa{function}\hlstd{(}\hlkwc{x}\hlstd{)} \hlkwd{sum}\hlstd{(}\hlopt{!}\hlkwd{is.na}\hlstd{(x)))}

  \hlstd{covs} \hlkwb{<-} \hlnum{NA}
  \hlstd{num.cats} \hlkwb{<-} \hlnum{NA}
  \hlcom{## covariates (only for lajeunesse)}
  \hlkwa{if}\hlstd{(lajeunesse) \{}
    \hlstd{num.covs} \hlkwb{<-} \hlkwd{length}\hlstd{(covariates)}
    \hlkwa{if}\hlstd{(num.covs} \hlopt{>} \hlnum{1}\hlstd{) \{}
      \hlstd{covs} \hlkwb{<-} \hlkwd{sapply}\hlstd{(}\hlkwd{get}\hlstd{(}\hlnum{2}\hlstd{), strsplit,} \hlkwc{split}\hlstd{=}\hlstr{':'}\hlstd{)}
      \hlstd{getCov} \hlkwb{<-} \hlkwa{function}\hlstd{(}\hlkwc{cov}\hlstd{)}
        \hlkwd{as.numeric}\hlstd{(}\hlkwd{as.factor}\hlstd{(}\hlkwd{sapply}\hlstd{(covs,} \hlkwa{function}\hlstd{(}\hlkwc{x}\hlstd{) x[cov])))}
      \hlstd{covs} \hlkwb{<-} \hlkwd{lapply}\hlstd{(}\hlnum{1}\hlopt{:}\hlstd{num.covs,} \hlkwa{function}\hlstd{(}\hlkwc{cov}\hlstd{)}
        \hlkwd{make.mat}\hlstd{(}\hlkwd{getCov}\hlstd{(cov)))}
      \hlkwd{names}\hlstd{(covs)} \hlkwb{<-} \hlstd{covariates}
      \hlstd{num.cats} \hlkwb{<-} \hlkwd{sapply}\hlstd{(covs, max,} \hlkwc{na.rm}\hlstd{=}\hlnum{TRUE}\hlstd{)}
    \hlstd{\}}
  \hlstd{\}}

  \hlkwd{return}\hlstd{(}\hlkwd{list}\hlstd{(}\hlkwc{Nstudy}\hlstd{=nstudy,}
              \hlkwc{Nyear}\hlstd{=nyear,}
              \hlkwc{Nobs}\hlstd{=nobs,}
              \hlkwc{V}\hlstd{=vi.mat,}
              \hlkwc{RR}\hlstd{=RR.mat,}
              \hlkwc{covariates}\hlstd{=covs,}
              \hlkwc{num.cats}\hlstd{=num.cats))}
\hlstd{\}}
\end{alltt}
\end{kframe}
\end{knitrout}

We then pacakges everything up in a function that also passes the JAGS
parameters (init values, thinning rate etc.) and runs the model.

\begin{knitrout}
\definecolor{shadecolor}{rgb}{0.969, 0.969, 0.969}\color{fgcolor}\begin{kframe}
\begin{alltt}
\hlcom{## takes preped data and a scale parameter and packages the data and}
\hlcom{## parameters for JAGS, then runs the analysis. Returns the JAGS}
\hlcom{## summary output}
\hlstd{runAnalysis} \hlkwb{<-} \hlkwa{function}\hlstd{(}\hlkwc{dd}\hlstd{,} \hlkwc{scale}\hlstd{) \{}

  \hlstd{my.inits} \hlkwb{<-} \hlkwa{function}\hlstd{() \{}
    \hlkwd{list}\hlstd{()}
  \hlstd{\}}
  \hlstd{my.params} \hlkwb{<-} \hlkwd{get.params}\hlstd{()}

  \hlstd{dd} \hlkwb{<-} \hlkwd{list}\hlstd{(}\hlkwc{data}\hlstd{=dd,} \hlkwc{inits}\hlstd{=my.inits,} \hlkwc{params}\hlstd{=}\hlkwd{get.params}\hlstd{())}

  \hlstd{bugs} \hlkwb{<-} \hlkwd{run.model}\hlstd{(dd,}
                    \hlkwc{n.thin}\hlstd{=scale,}
                    \hlkwc{n.iter}\hlstd{=(}\hlnum{1e4}\hlstd{)}\hlopt{*}\hlstd{scale,}
                    \hlkwc{n.burnin}\hlstd{=}\hlnum{1e1}\hlopt{*}\hlstd{scale,}
                    \hlkwc{n.chains}\hlstd{=}\hlnum{3}\hlstd{)}

  \hlkwd{return}\hlstd{(}\hlkwd{list}\hlstd{(}\hlkwc{data}\hlstd{=dd,}
       \hlkwc{bugs}\hlstd{=bugs,}
       \hlkwc{summary}\hlstd{=bugs}\hlopt{$}\hlstd{BUGSoutput}\hlopt{$}\hlstd{summary))}
\hlstd{\}}
\end{alltt}
\end{kframe}
\end{knitrout}

\begin{knitrout}
\definecolor{shadecolor}{rgb}{0.969, 0.969, 0.969}\color{fgcolor}\begin{kframe}
\begin{alltt}
\hlkwd{print}\hlstd{(out}\hlopt{$}\hlstd{bugs,} \hlkwc{dig}\hlstd{=}\hlnum{3}\hlstd{)}
\end{alltt}
\begin{verbatim}
## Inference for Bugs model at "model.jags", fit using jags,
##  3 chains, each with 1e+05 iterations (first 100 discarded), n.thin = 10
##  n.sims = 29970 iterations saved
##            mu.vect sd.vect      2.5%       25%       50%       75%
## cv.obs       1.154   0.134     0.909     1.061     1.149     1.241
## exp.mu       0.808   0.019     0.771     0.795     0.808     0.821
## sigma        0.189   0.023     0.145     0.172     0.188     0.204
## deviance -1293.537  36.474 -1362.740 -1318.528 -1294.309 -1269.421
##              97.5%  Rhat n.eff
## cv.obs       1.434 1.001  5100
## exp.mu       0.846 1.001 27000
## sigma        0.236 1.001  9400
## deviance -1219.532 1.001 20000
## 
## For each parameter, n.eff is a crude measure of effective sample size,
## and Rhat is the potential scale reduction factor (at convergence, Rhat=1).
## 
## DIC info (using the rule, pD = var(deviance)/2)
## pD = 665.1 and DIC = -628.4
## DIC is an estimate of expected predictive error (lower deviance is better).
\end{verbatim}
\end{kframe}
\end{knitrout}

exp.mu is the mean ratio of organic and conventional yields, which is
around $80.1\%$ with $95\%$ credible intervals ranging from around
$77--85\%$

Sigma is the between study variance, and cv.obs is the coefficient of
variation of $\gamma$ distribution the within-study variance
parameters are drawn from.


Values of Rhat < 1.1 indicate convergence, whcih we We can see from
inspecting the posteriors

\begin{knitrout}
\definecolor{shadecolor}{rgb}{0.969, 0.969, 0.969}\color{fgcolor}\begin{kframe}
\begin{alltt}
\hlkwd{ggs_density}\hlstd{(inspect.out)}
\end{alltt}
\end{kframe}
\includegraphics[width=\maxwidth]{figure/densityPlots-1} 

\end{knitrout}

The chains look wonderfully grassy

\begin{knitrout}
\definecolor{shadecolor}{rgb}{0.969, 0.969, 0.969}\color{fgcolor}\begin{kframe}
\begin{alltt}
\hlkwd{ggs_traceplot}\hlstd{(inspect.out)}
\end{alltt}
\end{kframe}
\includegraphics[width=\maxwidth]{figure/traceplots-1} 

\end{knitrout}



\section{Explanatory variables}
\label{sec:variables}
We also extended this model in order to accommodate analyses of study
characteristics such as crop type and management practices. We analyze
these additional explanatory variables one at a time because not all
studies reported all explanatory variables. In these analyses, for
cases where multiple organic treatments represented different
categories for a specific explanatory variable, they could not be
combined using Lajeunesse's method \citep{Lajeunesse2011}. The
potential bias resulting from non-independence of the response ratios
in these cases, however, would be minimized by the fact that they are
not pooled together in the analysis \citep{Lajeunesse2011}.

Letting $h$ index the categories for a particular explanatory variable
(e.g., crop species), we then have:
%
\begin{equation}
  \begin{split}
    \label{eq:fullModel_cov}
    y_{hijk} &=
    \mu + \gamma_h + \alpha_i + \eta_{ijk} +  \beta_{ij} + \epsilon_{ijk}\\
  \end{split}
\end{equation}
%

where $\gamma_h$ is the effect of the $h^{\mathrm{th}}$ category, and
the rest of the model parallels that given in Eq.~\ref{eq:fullModel}.

The model including explanatory variables in JAGS is

\begin{knitrout}
\definecolor{shadecolor}{rgb}{0.969, 0.969, 0.969}\color{fgcolor}\begin{kframe}
\begin{verbatim}
## model {
##     for(study in 1:Nstudy) {
##       for(year in 1:Nyear[study]) {
##         for(obs in 1:Nobs[study,year]) {
##           P[study,year,obs] <- 1/V[study,year,obs]
##           RR[study,year,obs] ~ dnorm(mu.RR[study,year,obs], P[study,year,obs])
##           mu.RR[study,year,obs] ~ dnorm(mu.study[study,year,obs],
##                                         tau.obs[study])
##           mu.study[study,year,obs] ~ dnorm(mu[cov[study,year,obs]], tau)
##         }
##       }
##       tau.obs[study] ~ dgamma(shape.obs, scale.obs)
##     }
## 
##     ## set priors on covariates
##     for(covs in 1:Ncov) {
##       mu[covs] ~ dnorm(0, 1E-3)
##       exp.mu[covs] <- exp(mu[covs])
##     }
##     tau <- 1 / (sigma * sigma)
##     sigma ~ dunif(0, 100)
## 
##     shape.obs <- (1/cv)^2
##     cv ~ dunif(0, 100)
##     scale.obs <- (1/in.scale)^2
##     in.scale ~ dunif(0, 100)
## 
##   }
\end{verbatim}
\end{kframe}
\end{knitrout}

Where the parameters names match the notation in
Equ.~\ref{eq:fullModel} and cov is a matrix of explanatory variables
assiciated with each response ratio.

To re-run the models with explanatory variables, we just added each
variable one-by-one to the covariates argument of the makeData()
function.

\bibliography{org_v_conv} \clearpage

\end{document}
