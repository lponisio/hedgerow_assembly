\documentclass[12pt]{article}
\usepackage[top=0.85in, bottom=0.93in, right=0.93in, left=0.93in,
paperwidth=8.5in, paperheight=11in, nohead]{geometry}
\geometry{letterpaper}
\usepackage[pdftex]{graphicx}
\usepackage{color}
\usepackage[normalem]{ulem}
\usepackage{amssymb}
\usepackage{amsmath}
\usepackage{epstopdf}
\usepackage{setspace}
\usepackage{mdwlist}
\usepackage{verbatim}
\usepackage[numbers]{natbib}

%% make bibliography more compact \setlength{\bibsep}{0in}
\renewcommand\bibsection{\subsubsection*{\refname}}

\begin{document}

\begin{centering}
  \large {\bf The assembly of pollinator communities in response to
    restoration} \\
\end{centering}
\vspace{0.15in}


\section{Goals}
\begin{enumerate}
\item Explore mechanisms underlying the assembly of plant-pollinator
  communities on on-farm habitat restorations by comparing empirical
  observations with assembly theory and adaptive foraging
\item How specialization changes in a pollination network during the
  pollinator guild assembly (colonization, persistence and extirpation
  processes over time)
\end{enumerate}


\section{Data prep}
\begin{enumerate}
\item starting plant community for each hedgerow summed over the first
  three years following installment
\item the number of pollinators at each site in the initial three
  years, and those added/lost after the initial three years
\item the mean, minimum and maximum pollinator species degree, summed
  over years/for each year
\end{enumerate}

\section{Model}
\begin{enumerate}
\item Use the initial network summed over years 1-3 for each hedgerow,
  randomly add species (equal to the number added in the observed
  data) and allow the community to assembly using adaptive foraging
\item  Keep track of
  \begin{enumerate}
  \item number of species that go extinct at each hedgerow
  \item the specialization of each plant and pollinator
  \end{enumerate}
\end{enumerate}

\section{Predictions}
\begin{enumerate}
\item ``generalist'' pollinators early in succession will become more
  specialized
\item ``generalist'' pollinators early in succession will visit a
  subset of what they originally visited, mainly the originally
  ``specialist'' plants
\end{enumerate}

\section{Considerations}
\begin{enumerate}
\item some plant species that were planted were never visited by
  pollinators in the first few years. This is likely because they did
  not produce many/any flowers/nectar resources
\item compare observed data to a model with/without adaptive foraging?
\item all the pollinator species needed (bees, syrphids, wasps
  etc.?). Not all have been identified to species.
\item weedy plant species ``added'' to communities by invasion, lost
  by death
\end{enumerate}


\end{document}

%%% Local Variables:
%%% mode: latex
%%% TeX-PDF-mode: t
%%% End:


