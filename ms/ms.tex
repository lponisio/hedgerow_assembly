\documentclass[12pt]{article} 
\usepackage[pdftex]{graphicx}
\usepackage{natbib} 
\usepackage{color}
\usepackage{amsmath} 
\usepackage{amssymb} 
\usepackage{verbatim}
\usepackage{mathpazo} 
\usepackage{setspace}
\usepackage{multirow}
\usepackage{fullpage}
\usepackage{lscape}
\usepackage{fancyhdr}
\usepackage[normalem]{ulem} 
\usepackage{hyperref}
\usepackage[parfill]{parskip}
\usepackage{xr}
\usepackage{paralist}
\usepackage{xr}
\externaldocument{SI}
\hypersetup{colorlinks=true, linkcolor=black, citecolor=black}
\RequirePackage{lineno}

\newcommand{\flagged}[1] {
  \textcolor{blue}{#1}
}

\def\title{The temporal assembly of plant-pollinator networks
  following restoration}
\def\author{Lauren C.\ Ponisio$^1$, Marilia P. Gaiarsa$^2$, Claire Kremen$^1$}

\def\runninghead{Coevolution and network structure}
\def\keywords{evolution, nestedness, modularity, bipartite}

\def\extras{
  \begin{itemize}
  \item Submitted as a Letter
  \item Abstract word count: 
  \item Main text word count: 
  \item Number of references: 
  \item Number of figures:
  \end{itemize}
}

\def\affiliation{
  \begin{enumerate}
  \item Department of Environmental Science, Policy, and Management\\
    University of California, Berkeley\\
    130 Mulford Hall\\
    Berkeley, California, USA\\
    94720\\
  \item Departamento de Ecologia\\
    Universidade de Sao Paulo\\
    Sao Paulo, SP, Brazil\\
    05508-900\\
  \end{enumerate}
}

\newcommand{\mstitlepage}{
  \paragraph{Running head:} \textsc{\runninghead}
  % \parindent=0pt
  \begin{center}%
    {\LARGE \title \par}%
    \vskip 3em%
    {\large
      \lineskip .75em%
      \begin{tabular}[t]{c}%
        \author
      \end{tabular}\par}%
    \vskip 1.5em%
  \end{center}\par
  \affiliation
}
\clearpage

\begin{document}

\mstitlepage
\doublespacing
\linenumbers
\clearpage

\begin{abstract}
  The structure of networks is related to ability of communities to
  maintain function in the face of species extinction. Understanding
  network structure and how it relates to network disassembly,
  therefore, is a priority for system-level conservation biology.  We
  explore the assembly of plant-pollinator communities on native plant
  restorations in the Central Valley of California. The assembling
  communities are paired with unrestored field margins (controls) and
  mature (non-assembling) hedgerows. We determine whether there are
  change points in the assembly of the communities where the network
  undergoes significant reorganization. We are also ask how are the
  individual species changing their interaction patterns? What does
  this mean for the topology/resilience of the network? We also
  attempted to adapt a financial model to mutualistic networks. Our
  biggest difficulty with this approach was to translate the price
  term to mutualistic systems. We explored a range of approaches, such
  as number of visits a species performs. However, it seems that
  financial systems cannot be easily translated to mutualistic
  systems. In addition, we used a Changing Point Detection Algorithm
  to assess weather the different communities went through a critical
  reorganization on their interaction patterns. We were able to
  identify some changing points in the communities, and also to
  explore some general patterns commonly used to describe ecological
  networks. For example, on the network level, networks become
  increasingly modular and less nested, whereas on the species level,
  species become more specialized, as resources become more reliable.
\end{abstract}

Keywords: changing points, temporal networks, hedgerows, species
interactions

\clearpage

\section*{Introduction}
\label{sec:introduction}
\begin{itemize}
\item The structure of networks is related to ability of communities
  to maintain function in the face of species extinction.
\item A key restoration aim is to facilitate assembly of robust
  networks; thus it is critical to study how restoration
  influences the assembly of plant-pollinator interactions.
\item few theories about how networks assemble, preferential
  attachment 
\item To date, only two field studies have examined how networks
  assemble over time, often using space for time gradients.
\item \cite{Olesen2008} was investigated day-to-day, temporal assmebly
  of a plant-pollinator network within a season, taking advantage of
  the extreme seasonality of pollinator communities in Greenland.
  \cite{Olesen2008} found that within a season, the network assembly
  was similar to preferential attachment. New species tended to
  interact with already well-connected species, likely because these
  species are either more abundant or more temporally persistent.
\item Studying primary succession along a glacier foreland,
  \cite{albrecht2010plant} found a similar pattern where nestedness, a
  pattern of interactions where a generalist core interacts with both
  specialist and generalist species, increased as the community
  aged. % Studying ``managed succession'' of a clear-cut pine forest,
  % \cite{devoto2012understanding} found changes in network structure
  % were explained by a combination of age, tree density and variation
  % in tree diameter. 
\item Even non-succesional temporal dyanmics suggest a stable core of
  generalsits persiss despite high turnover of peripheral species
  \citep{fang2012relative, diaz2010changes, alarcon2008year}.
\item In contrast to the ordered netowrk build-up described by
  perferential attachment, assembly may be punctated by significant
  reorganizations of interactions. For example, as new species are
  added, resident species change their interaction parteners to
  minimize competition, or become extinct.
\end{itemize}


\section*{Materials \& Methods}
\label{sec:methods}

\subsection*{Study sites and collection methods}
\label{sec:study-sites}

\section*{Results}
\label{sec:results}

\section*{Discussion}
\label{sec:discussion}

\section*{Acknowledgments}
\label{sec:acknowledge}

We would like to thank Leto Peel and Aaron Clauset for their
invaluable discussions and for help with the change point analysis.
We thank the growers and land owners that allowed us to work on their
property.  We also appreciate the identification assistance of expert
taxonomists Martin Hauser, Robbin Thorp and Jason Gibbs.  This work
was supported by funding from the Army Research Office
(W911NF-11-1-0361 to CK), the Natural Resources Conservation Service
(CIG-69-3A75-12-253, CIG-69-3A75-9-142, CIG-68-9104-6-101 and
WLF-69-7482-6-277 to The Xerces Society), the National Science
Foundation (DEB-0919128 to CK), The U.S.  Department of Agriculture
(USDA-NIFA 2012-51181-20105 to Michigan State University).  Funding
for LCP was provided by an NSF Graduate Research Fellowship and the
USDA NIFA Graduate Fellowship. FUNING FOR MARILLIA.

\bibliographystyle{ecol_let}

\bibliography{refs}


\end{document}

%%% Local Variables:
%%% mode: latex
%%% TeX-PDF-mode: t
%%% End:
