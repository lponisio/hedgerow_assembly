\documentclass[12pt]{article}
\usepackage[pdftex]{graphicx}
\usepackage[numbers]{natbib}
% \usepackage{natbib}
\usepackage{color}
\usepackage{amsmath}
\usepackage{amssymb}
\usepackage{verbatim}
\usepackage{mathpazo}
\usepackage{setspace}
\usepackage{multirow}
\usepackage{fullpage}
\usepackage{lscape}
\usepackage{array}
\usepackage{paralist}
\usepackage{fancyhdr}
\usepackage{graphicx}
\usepackage{longtable}
\usepackage{times}
\usepackage{textcomp}
\usepackage{xr}
\usepackage{etoolbox}
\usepackage{filecontents}
\usepackage{url}
\externaldocument{SI}

\makeatletter
\def\@biblabel#1{#1.\\}
\makeatother

\bibliographystyle{gcb}

\parindent=2em
\setlength{\parskip}{1ex}

\RequirePackage{lineno}


\def\title{The assembly of plant-pollinator networks}

\def\author{Lauren C.\ Ponisio$^{1}$, Marilia P.\ Gaiarsa$^2$
  and Asher Mullokandov$^3$}

\def\runninghead{Temporal network assembly}

\def\affiliation{
  \begin{enumerate}
  \item Department of Environmental Science, Policy, and Management\\
    University of California, Berkeley\\
    130 Mulford Hall\\
    Berkeley, California, USA\\
    94720\\
  \item Departamento de Ecologia\\
    Universidade de Sao Paulo\\
    Sao Paulo, SP, Brazil\\
    05508-900\\
  \item Department of Physics\\
    Boston University\\
    Boston, MA, USA\\
    02215\\
  \end{enumerate}
}

\newcommand{\mstitlepage}{
  \paragraph{Running head:} \textsc{\runninghead}
  % \parindent=0pt
  \begin{center}%
    {\LARGE \title \par}%
    \vskip 3em%
    {\large
      \lineskip .75em%
      \begin{tabular}[t]{c}%
        \author
      \end{tabular}\par}%
    \vskip 1.5em%
  \end{center}\par
  \affiliation
}
\clearpage

\begin{document}

\mstitlepage
\doublespacing
\linenumbers
\clearpage

\begin{abstract}
  The structure of networks is related to ability of communities to
  maintain function in the face of species extinction. Understanding
  network structure and how it relates to network disassembly,
  therefore, is a priority for system-level conservation biology.  We
  explore the assembly of plant-pollinator communities on native plant
  restorations in the Central Valley of California. The assembling
  communities are paired with unrestored field margins (controls) and
  mature (non-assembling) hedgerows. We determine whether there are
  change points in the assembly of the communities where the network
  undergoes significant reorganization. We are also ask how are the
  individual species changing their interaction patterns? What does
  this mean for the topology/resilience of the network? We also
  attempted to adapt a financial model to mutualistic networks. Our
  biggest difficulty with this approach was to translate the price
  term to mutualistic systems. We explored a range of approaches, such
  as number of visits a species performs. However, it seems that
  financial systems cannot be easily translated to mutualistic
  systems. In addition, we used a Changing Point Detection Algorithm
  to assess weather the different communities went through a critical
  reorganization on their interaction patterns. We were able to
  identify some changing points in the communities, and also to
  explore some general patterns commonly used to describe ecological
  networks. For example, on the network level, networks become
  increasingly modular and less nested, whereas on the species level,
  species become more specialized, as resources become more reliable.
\end{abstract}

Keywords: pollinators, community assembly, networks, changing points,
temporal networks
 \clearpage

\end{document}

%%% Local Variables:
%%% mode: latex
%%% TeX-PDF-mode: t
%%% End:
