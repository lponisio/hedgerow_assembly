\documentclass[12pt]{letter}
\usepackage[margin=2.5cm]{geometry}
\usepackage{mathpazo}
\usepackage[pdftex]{graphicx}
\usepackage{color}
\usepackage[normalem]{ulem}
\usepackage{times}
\usepackage{textcomp}

\address{Dept of Environmental Science,\\
  Policy, and Management\\
  130 Mulford Hall\\
  University of California, Berkeley\\
  Berkeley, California, 94720, USA} \raggedright
\begin{document}

\begin{letter}{}

  \opening{Dear editor,}

  W are pleased to submit our manuscript \textit{Major interaction
    reorganizations punctuate the assembly of pollination networks}
  for consideration as a Letter in \textit{Ecology Letters}.

  As the world continues to lose species at an alarming rate, it has
  become increasingly imperative to aid the recovery of lost
  interactions and component biodiversity through ecological
  restoration. When a species goes extinct, its interactions are also
  lost. We know little, however, about how to re-assemble interacting
  communities through restoration, or the process of ecological
  network assembly more generally. Our research deals with two
  fundamental aspects of the ecological theory that underlie
  restoration ecology: understanding how species-rich communities
  assemble, and how these assemblages change through time. Our work
  uses a novel method to examine the temporal changes in networks,
  represents the first long-term study of temporal assembly of
  ecological networks, and challenges the currently accepted community
  assembly theory of preferential attachment.

  Preferential attachment predicts that species entering a network are
  more likely to interact with species that are already well-connected
  --- and is well-supported in the network literature to date.  We
  analyzed plant-pollinator interaction data comprising eight years
  and \texttildelow $20,000$ records at native plant restorations in
  the Central Valley of California. We find that species are highly
  dynamic in their network position, causing community assembly to be
  punctuated by major interaction reorganizations. The most persistent
  and generalized species are also the most variable in their network
  positions, challenging what is expected through preferential
  attachment theory. Our results are compelling and provide empirical
  evidence that fundamentally alter our understanding of how
  communities assembly and how species interactions change through
  time. Our insights will help to inform effrots to re-assembly robust
  communities through restoration.  We believe that these exciting
  results linking three major ecological fields (interaction networks,
  community assembly dynamics and restoration ecology) will be of
  broad interest to the readership of \textit{Ecology Letters}.


  % Our manuscript is original and was carried out fully by the authors.
  % All authors agree with the contents of the manuscript.  This
  % manuscript is not published, nor is it in consideration for
  % publication elsewhere.  All research not of the authors' is fully
  % acknowledged.  The authors declare no conflict of interest. All
  % appropriate ethical standards were followed. 
  Thank you for reviewing our manuscript and we hope you will find it
  suitable for consideration in \textit{Ecology Letters}.
  
  Regards,
  Lauren C. Ponisio, PhD\\
  Claire Kremen, Professor
%is there a reason why I am not here? Just curious =0) 
\end{letter}
\end{document}

%%% Local Variables:
%%% mode: latex
%%% TeX-PDF-mode: t
%%% End:
